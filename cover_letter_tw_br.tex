%!TEX TS-program = xelatex
\documentclass[]{friggeri-cover-letter}
\usepackage[brazil,english]{babel}
\usepackage{xltxtra}
\defaultfontfeatures{Ligatures=TeX}
\usepackage{afterpage}
\usepackage{hyperref}
\usepackage{hyphenat}
\usepackage{color}
\usepackage{xcolor}
\usepackage{fancyhdr}
\usepackage{multilanguage}

\setdoclang{br}{brazil}
%\setdoclang{en}{english}

\hypersetup{%
    colorlinks,
    linkcolor={red!50!black},
    citecolor={blue!50!black},
    urlcolor={blue!80!black},
    pdfborder={0 0 0},
}

\pagestyle{fancy}
\lhead{}
\chead{}
\rhead{}
\rfoot{{%
    \footnotesize{%
        \langif{br}{%
        %lang pt-br
            Veja a última versão deste documento em:
            \href{https://github.com/diraol/cv/releases/}{https://github.com/diraol/cv/releases/}%
        }{%
        %lang en
            See the lastest version of this document at:
            \href{https://github.com/diraol/cv/releases/}{https://github.com/diraol/cv/releases/}%
        }
}}}
\renewcommand{\headrulewidth}{0pt}
\renewcommand{\footrulewidth}{0pt}

\hypersetup{%
    pdftitle={Diego Rabatone Oliveira},
    pdfauthor={Diego Rabatone Oliveira},
    pdfsubject={Carta de Apresentação/Cover Letter},
    pdfkeywords={Diego Rabatone Oliveira, diraol},
    colorlinks=false,       % no lik border color
    allbordercolors=white    % white border color for all
}
%\addbibresource{bibliography.bib}
\RequirePackage{xcolor}
\definecolor{pblue}{HTML}{0395DE}

\begin{document}
\thispagestyle{empty}
\pagenumbering{gobble}% Remove page numbers (and reset to 1)
%%%%%%%%%%%%%%%%%%%%%%%%%%%%%%%%%%%%%%%%%%
%\langif{br}{%
%% text for lang pt-br
%
%}{%
%% text for lang pt-br
%
%}
%\langif{br}{}{}
%%%%%%%%%%%%%%%%%%%%%%%%%%%%%%%%%%%%%%%%%%
%header{name}{lastname}{title}{city}{country}{phone}{email}{homepage}
\header{Diego}{Rabatone O.}{\langif{br}{Engenheiro de Computação}{Computer Engineer}}{São Paulo}{\langif{br}{Brasil}{Brazil}}{+55~(11)~9~8231~4249}{diraol@diraol.eng.br}{http://cv.diraol.eng.br}

\textbf{Para: Thoughtworks Brasil}\\
\textbf{Sobre:} Candidatura para Vagas de Desenvolvedoras/es Consultoras/es de Software - Brasil

\vfill

Prezadas/os,

\vfill

me candidato à vaga de "Desenvolvedoras/es Consultoras/es de Software" junto à ThoughtWorks Brasil, após ter tomado ciência da abertura de vagas após divulgação realizada pelo Luciano Ramalho em se twitter.

Tenho grande interesse em me juntar ao time da Thoughtworks Brasil tanto por sua reconhecida qualidade técnica, quanto por sua singular visão de mundo, inclusiva e transformadora.

Há mais de 6 anos atuo em movimentos sociais com o objetivo de causar transformações sociais, geralmente utilizando a tecnologia como meio e ferramenta. Enquanto membro da comunidade Transparência Hacker, atuei na elaboração de projetos de lei (Lei de Acesso à Informação e Marco Civil da Internet, dentre outras), participei de diversos eventos e hackahtons, tendo recebido 5 prêmios por projetos ligados a estas atividades, e também participei de diversas atividades de formação, como a Rodada Hacker - oficinas de programação para empoderamento de mulheres.

De forma costumeira, quando utilizo softwares livres para os projetos que participo, costumo tentar colaborar com o projeto, seja com contribuições no código, seja com documentação ou colaboração na comunidade por meio de respostas a dúvidas que outras pessoas tenham.

Por conta de minha formação de engenheiro e de minhas formações complementares, busco sempre avaliar os problemas com um olhar sistêmico, entendendo quais são os stakeholders envolvidos, direta e indiretamente, 

juntar ao Advocating for Free Software, as in Freedom, is one of the main objectives of my professional and activist life.
That was the main reason that led me to found a Free Software Group on the Univerisity. I have been using, producing and contributing to Free Software since 2007 and it was always prerequisite on my jobs and personal projects, and everything I have recently produced is on my recent jobs are on my github account.

I have started my Web Developer carrer in 2002, while in school, learning mostly on web forums. My engineering course prepared me to the challanges of technological carrer, in which there is a constant need for learning new technologies, and it also thought me the relevance of multidisciplinary knowledges working togheter.

My main experience is on the web field. I have been using Javascript, HTML, CSS, Drupal, Wordpress, Django and Mediwiki, MySQL and PostgreSQL but I have also some experience as SysAdmin, both as professional and activist.
Among my acitivies on the University Free Software Group (\href{http://polignu.org}{PoliGNU} - \url{http://polignu.org}), I taught Gimp, Inkscape and LaTeX, organized Install Fests and promoted workshops about Free Software philosophy to the community.

Besides my strong independent profile, I enjoy and fi well working within a team or a community. I feel glas to contribute on management and strategic decisions when required. I am able to take on the responsibility of this position immediately, and have the enthusiasm and determination to ensure that I make a success of it. Work at FSF would be a life objective achievement.

Thank you for taking the time to consider this application and I look forward to hearing from you in the near future.

\vfill

\hfill Yours sincerely,

\hfill \textbf{Diego Rabatone Oliveira}

\vfill

\footnotesize{\thinfont\color{lightgray}\textit{Attached: curriculum vitae - also found at:} \textit{\url{http://github.com/diraol/cv/releases}}}

\end{document}


















Eu entendo que, das opções postas, a "Promoção da equidade de oportunidades no ambiente de trabalho"  é a mais sensível de todas.
Isto pois pensar profundamente nesta problemática envolve pensar na interseccionalidade das diversas opressões e discriminações que temos (social, econômica, racial, de gênero, opção sexual, etc). Além disso, a equidade de oportunidades no ambiente de trabalho acaba por envolver um número maior de stakeholders da empresa, desde colegas de trabalho (mesmo nível hierárquico) até gestores e tomadores de decisão, que precisam ser sensibilizados para tais questões e precisam refletir e observá-las cotidianamente.
Hoje observa-se, de forma ainda incipiente, que pessoas pertencentes a grupos socialmente desfavorecidos (mulheres, indígenas, negro/as, pessoas trans, pessoas com deficiências físicas, etc) conseguiram uma maior inserção no mercado de trabalho, mas ainda continuam a sofrer discriminações, claras ou veladas, e a terem menos oportunidades efetivas. Um exemplo é que continuamos a observar grande diferenciação no salário entre homens e mulheres, negra/os e branca/os, para uma mesma posição, além de ser clara também a maior dificuldade de ascensão na carreira.
Uma possível forma de superação deste problema, dentre muitas outras, é o estabelecimento de metas de diversidade, não apenas no ingresso na empresa, mas também em termos de ascensão de carreira, o estabelecimento de metas para diretorias, coordenações, etc.